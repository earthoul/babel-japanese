%#!lualatex
\documentclass[lualatex,a4paper,11pt]{jlreq}
\usepackage{luatexja-fontspec}

%%%%%%%%%%%%%%%%
% additional packages
\usepackage{amsmath}
\usepackage{array}
\usepackage[defaultsups]{newpxtext}
\usepackage[zerostyle=c,straightquotes]{newtxtt}
\usepackage{newpxmath}
\usepackage[svgnames]{xcolor}
\usepackage[hyperfootnotes=false]{hyperref}
\usepackage{hologo}
\usepackage{xspace}

\def\Pkg#1{\textsf{#1}}
\def\Opt#1{\texttt{#1}}
\def\ClsY#1{\Pkg{\textcolor{Black}{#1}}}
\def\ClsT#1{\Pkg{\textcolor{Green}{#1}}}
\def\ClsYT#1{\Pkg{\textcolor{Blue}{#1}}}
\def\file#1{\textsf{#1}}
\def\code#1{\texttt{#1}}
\def\note#1{{\small #1}}

\usepackage[japanese,english]{babel}% References in English

\usepackage{listings}
\lstset{%
  language=[LaTeX]{TeX},
  basicstyle={\small\ttfamily},
  texcsstyle={*\color{blue}},
  commentstyle={\color{Green}},
  columns=[l]{fullflexible},
  numbers=left,
  xrightmargin=0\zw,
  xleftmargin=3\zw,
  aboveskip=3pt,
  belowskip=3pt,
  moretexcs={setlength,setmainfont}
}

\usepackage{shortvrb}
\MakeShortVerb*{|}
%%%%%%%%%%%%%%%%

% logos
\def\eTeX{$\varepsilon$-\TeX}
\def\pTeX{p\kern-.10em\TeX}
\def\upTeX{u\pTeX}
\def\epTeX{$\varepsilon$-\pTeX}
\def\eupTeX{$\varepsilon$-\upTeX}
\def\pLaTeX{p\LaTeX}\def\pLaTeXe{p\LaTeXe}
\def\upLaTeX{u\pLaTeX}\def\upLaTeXe{u\pLaTeXe}
\def\pdfTeX{\hologo{pdfTeX}}
\def\pdfLaTeX{\hologo{pdfLaTeX}}
\def\XeTeX{\hologo{XeTeX}}
\def\XeLaTeX{\hologo{XeLaTeX}}
\def\LuaTeX{\hologo{LuaTeX}}
\def\LuaLaTeX{\hologo{LuaLaTeX}}

\def\TL{\TeX\ Live\xspace}
\def\JTDC{Japanese \TeX\ Development Community}

\def\_{\leavevmode\vrule width .45em height -.2ex depth .3ex\relax}

\frenchspacing
\begin{document}

\title{\textsf{\textbf{Guide to Japanese typesetting with \LaTeX}}}
\author{\JTDC\null
\thanks{\url{https://texjp.org},\ e-mail: \texttt{issue(at)texjp.org}}}
\maketitle

% 日本語組版は西欧言語とは全然違う
Typesetting requirements of Japanese are completely different from
those of Western languages, so setting parameter values and additional
hyphenation patterns are not enough.

% babel-japanese (japanese.ldf) は訳語を登録するだけ
% 日本語組版に必要な設定は行わない
The package \Pkg{babel-japanese} provides ``\Opt{japanese}''
option for Babel package.
However, it only defines translated replacement texts for
keywords and dates (e.g. ``\contentsname'' $\rightarrow$
``{\selectlanguage{japanese}\contentsname}'',
\verb+\today+ $\rightarrow$ ``{\selectlanguage{japanese}\和暦\today}'')
and it does not care about any Japanese typesetting requirements.

% この文書の目的:ちゃんと日本語する簡単な方法の紹介
%-- 日本語を「ちゃんと」組むことが西欧言語とは全く異なることを伝える。
%-- ほんの数文字・数行の行組版だけでも組版規則が全く異なる。
%-- ページレイアウトに関わる設定値(段落字下げ,行長,行送りなど)が
%-- 異なるのは,想像の範囲内と思われる。
This document describes how to get an ``acceptable'' Japanese
typesetting result and the eminent differences between Western
and Japanese typesetting requirements.


% とりあえず LuaLaTeX + \usepackage{luatexja} オススメ
\section{Short introduction: using \LuaLaTeX}

To obtain a properly formatted Japanese document, the easiest way
is to use \LuaLaTeX\ with Lua\TeX-ja package.

\begin{itemize}
  \item Using the \Pkg{jlreq} class:
\begin{lstlisting}
  \documentclass[book]{jlreq}
  \begin{document}
  \chapter{日本語の文書}% Japanese document
  こんにちは,日本。% Hello, Japan.
  \end{document}
\end{lstlisting}
    Save the above as \file{jp1.tex} and run the following command.
\begin{verbatim}
    $ lualatex jp1
\end{verbatim}
    You will get \file{jp1.pdf} which is properly formatted and
    a Japanese font family ``HaranoAji Mincho/Gothic'' is embedded.
 \item If you prefer classes not for Japanese,
    you can load \Pkg{luatexja} directly:
\begin{lstlisting}
  \documentclass{book}
  \usepackage{luatexja}
  \begin{document}
  ...
\end{lstlisting}
    * Note that using non-Japanese classes will lead to unnatural
    page layouts for Japanese books, especially regarding line gaps
    and line lengths.
\end{itemize}


% 他の選択肢:upLaTeX が先
\section{Relatively stable alternative: \upLaTeX}

\LuaLaTeX\ is relatively new,
and Lua\TeX-ja is under active development.
As a more stable alternative, you can use \upLaTeX.
It is a somewhat legacy implementation, which requires TFM files
for typesetting Japanese characters (called JFM).
If you use only two fixed-width Japanese fonts, using a standard
class is enough:
\begin{lstlisting}
  \documentclass[uplatex]{jsbook}% or, \documentclass{ujbook}
  \begin{document}
  ...
\end{lstlisting}
Save the above as \file{jp2.tex} and run the following command.
\begin{verbatim}
    $ uplatex jp2
\end{verbatim}
You will get \file{jp2.dvi}.\bigskip

\upLaTeX\ and \pLaTeX\ (explained later) are often referred to
as (u)\pLaTeX. They always output DVI files.
To convert DVI containing Japanese characters into PDF,
the easiest way is to use dvipdfmx.
If you have \file{jp2.dvi}, running the following command
\begin{verbatim}
    $ dvipdfmx jp2
\end{verbatim}
will generate \file{jp2.pdf}.\bigskip

Since the default DVI driver in \TL is set to
\Opt{dvips},\footnote{Though dvips supports DVI with Japanese fonts,
additional settings for Ghostscript are required to convert
the resulting PostScript file into PDF.}
it is recommended to add a global driver option
to be passed to all driver-dependent packages
(e.g. \Pkg{graphicx}, \Pkg{color}, \Pkg{hyperref}):
\begin{lstlisting}
  \documentclass[uplatex,dvipdfmx]{jsbook}
  \usepackage{graphicx}
  \begin{document}
  ...
\end{lstlisting}


% レガシーな pLaTeX は最後
\section{Using legacy \pLaTeX}

Some old Japanese journal classes may require \pLaTeX.
It is a legacy implementation which was born in 1980s,
before the release of Unicode 1.0.
\pLaTeX\ only supports a limited character set, JIS~X~0208
(6879 characters).
\begin{lstlisting}
  \documentclass{jsbook}% or, \documentclass{jbook}
  \begin{document}
  ...
\end{lstlisting}
Save the above as \file{jp3.tex} and run the following command.
\begin{verbatim}
    $ platex jp3
\end{verbatim}
You will get \file{jp3.dvi}.
Again, the DVI file can be converted into PDF using dvipdfmx.


%% ちゃんと日本語を組むには?
%-- 禁則文字を除く任意箇所での行分割が可能
%-- 和文文字直後の改行文字は空白にならない
%-- 約物どうしの字間調整
%-- 和欧文間空白の挿入
\section{A short summary of differences between Western and Japanese typesetting}

The sections above described some practical ways to achieve
``acceptable'' Japanese typesetting results.
So, what are the minimum requirements for ``acceptable'' results
for native Japanese readers?

The W3C Working Group Note ``Requirements for Japanese Text Layout''
\cite{w3c-jlreq} is one of the comprehensive documentation which
describes issues of text composition in the Japanese writing system.
However, they are too complicated to explain here thoroughly,\footnote{They
include specification of basic page layout (\emph{kihon-hanmen}; 基本版面),
vertical alignment of Japanese text (\emph{tate-gumi}; 縦組),
small-sized supplementary text (\emph{ruby}; ルビ), inline cutting note
(\emph{warichu}; 割注) etc.} so we focus on the main issues regarding
simple and short horizontal writing within a few lines.

\newbox\JEXBOX % stores example output
\newbox\JJgridbox
\def\JJsq{%
  \hbox to 1\zw{\hss\fboxsep=-.5\fboxrule\fbox{%
   \hskip\dimexpr-.5\zw-.2pt\vrule width.4pt height.08\zw depth.12\zw%
  \hskip\dimexpr.5\zw-.2pt\relax}\hss}}
\def\JJsb{\hbox to 1\zw{\hss\fboxsep=-.5\fboxrule\fbox{%
  ■\hskip\dimexpr-.5\zw-.2pt\vrule width.4pt height.08\zw depth.12\zw%
  \hskip\dimexpr.5\zw-.2pt\relax}\hss}}
\setbox\JJgridbox=\hbox to 12\zw{%
  \JJsq\JJsq\JJsq\JJsq\JJsb\JJsq\JJsq\JJsq\JJsq\JJsb\JJsq\JJsq}
\def\JJoutbox#1{{%
  \ltjsetparameter{kanjiskip=0pt plus .2\zw minus .2\zw}%
  \hbox{\textcolor{cyan}{\copy\JJgridbox}\hskip-12\zw%
  \vbox{\hsize=12\zw\noindent\leftskip0pt\rightskip0pt\parfillskip0pt#1\hss}\vrule}%
}}

Consider the following input, Listing \ref{list:src1}. Using \LuaLaTeX,
it is allowed to include any Unicode character out-of-the-box,
though a suitable Japanese font should be declared.

\begingroup
\begin{figure}[tbp]
\ltjsetparameter{kanjiskip=0pt,xkanjiskip=0pt}
\begin{lstlisting}[caption={Example input using \LuaLaTeX},label={list:src1}]
  %#!lualatex
  \documentclass{article}
  \usepackage{fontspec}
  \setmainfont{HaranoAjiMincho-Regular}
  \setlength{\textwidth}{250pt}
  \setlength{\overfullrule}{5pt}
  \begin{document}
  \noindent
  % first line = exactly 250pt (= 10pt x 25)
  ◆◆◆◆◆◆◆◆◆◆◆◆◆◆◆◆◆◆◆◆◆◆◆◆◆\\
  改行後
  空白が入る。日本語が連続すると改行できず、はみ出す。\\
  句読点と、(括弧)の間隔が広い。%
  日本語とLatinの間が空かない。
  \end{document}
\end{lstlisting}
\end{figure}
\endgroup

Figure \ref{fig:src1} shows the exact output from the input above.
It is not acceptable, especially for the following reasons:
\begin{itemize}
  \item Japanese text has no interword spaces, but
    line break is allowed between any Japanese characters
    except before some punctuation marks (\emph{kinsoku}; 禁則).
    The overfull rules that appeared in Figure \ref{fig:src1} come
    from not knowing such line breaking rules.
  \item Similarly, it is common to insert a new line character after
    arbitrary Japanese characters in the input source.
    Such a new line character should not be treated as a blank space.\\
    Wrong: \JJoutbox{改行後 空白が入る。…\hfill}\quad
    Correct: \JJoutbox{改行後空白が入る。…\hfill}
  \item Ordinary Japanese characters are designed to have virtual
    full-width (\emph{zenkaku}) bounding box. However, spacing
    between adjacent punctuation marks must be truncated.\\
    Wrong: \JJoutbox{句読点と、\null (括弧)の間…\hfill}\quad
    Correct: \JJoutbox{句読点と、(括弧)の間…\hfill}
  \item It is preferred to insert a quarter em space (\emph{shibu aki})
    between Japanese and Latin characters.\footnote{Such a space
    is often called ``xkanjiskip'' glue in Japanese \TeX\ world.}\\
    Wrong: \JJoutbox{日本語と\null Latin\null の間…\hfill}\quad
    Correct: \JJoutbox{日本語とLatinの間…\hfill}
\end{itemize}
Figure \ref{fig:src1-mod} shows an acceptable output of the same text.
You will see that choosing a Japanese font on Unicode-enabled engine
is not enough for obtaining properly formatted result.

All of the methods explained in the previous sections,
\begin{enumerate}
  \item \LuaLaTeX\ with Lua\TeX-ja package,
  \item \upLaTeX,
  \item \pLaTeX,
\end{enumerate}
are developed to support (most of) the requirements mentioned above,
and their output is most likely to be acceptable.
Therefore, we recommend to use one of those platforms.

\begingroup
\ltjsetparameter{jacharrange={-1,-2,-3,-4,-5,-6,-7,-8,-9}}
\setmainfont{HaranoAjiMincho-Regular}
\global\setbox\JEXBOX=\vbox{\hsize=250pt\fontsize{10pt}{12pt}\selectfont
\setlength{\overfullrule}{5pt}\noindent
  ◆◆◆◆◆◆◆◆◆◆◆◆◆◆◆◆◆◆◆◆◆◆◆◆◆\\
  改行後
  空白が入る。日本語が連続すると改行できず、はみ出す。\\
  句読点と、(括弧)の間隔が広い。%
  日本語とLatinの間が空かない。}
\endgroup

\begin{figure}[tbp]
\centering\leavevmode\fbox{\box\JEXBOX}
\caption{Output from Listing \ref{list:src1} (not acceptable)}
\label{fig:src1}
\end{figure}

\begingroup
\setmainfont{HaranoAjiMincho-Regular}
\global\setbox\JEXBOX=\vbox{\hsize=250pt\fontsize{10pt}{12pt}\selectfont
\setlength{\overfullrule}{5pt}\noindent
  ◆◆◆◆◆◆◆◆◆◆◆◆◆◆◆◆◆◆◆◆◆◆◆◆◆\\
  改行後
  空白が入る。日本語が連続すると改行できず、はみ出す。\\
  句読点と、(括弧)の間隔が広い。%
  日本語とLatinの間が空かない。}
\endgroup

\begin{figure}[tbp]
\centering\leavevmode\fbox{\box\JEXBOX}
\caption{Acceptable output of the above text}
\label{fig:src1-mod}
\end{figure}


%% 各種文書への導線をはる
\section{Further readings}

\begin{itemize}
  \item ``The Lua\TeX-ja package'' (\file{luatexja-en.pdf}, in English)
    \begin{itemize}
      \item For beginners, refer to Part I ``User's manual.''
      \item For more interested users, refer to Part II ``Reference.''
      \item For developers, refer to Part III ``Implementations.''
      \item Japanese edition: \file{luatexja-ja.pdf}
    \end{itemize}
  \item ``Guide to \pTeX\ for developers unfamiliar with Japanese''
        (\file{ptex-guide-en.pdf}, in English)
    \begin{itemize}
      \item For developers who aim to support
        \pTeX/\pLaTeX\ and its variants \upTeX/\upLaTeX.
    \end{itemize}
  \item ``About \upLaTeXe''
        (\file{uplatex-en.pdf}, in English)
    \begin{itemize}
      \item For users of \upLaTeX, but a somewhat legacy document.
      \item Japanese edition: \file{uplatex-en.pdf}
    \end{itemize}
  \item ``About \pLaTeXe''
        (\file{platex-en.pdf}, in English)
    \begin{itemize}
      \item For users of \pLaTeX, but a somewhat legacy document.
      \item Japanese edition: \file{platex-en.pdf}
    \end{itemize}
  \item ``Babel-Option \Opt{japanese}''
        (\file{babel-japanese.pdf}, in both English and Japanese)
    \begin{itemize}
      \item Basic introduction of \Opt{japanese} option of \Pkg{babel}.
      \item Provides translated replacement texts for keywords and date.
      \item Note: the translations provided by this package may not
        become effective on most of Western classes, due to lack of
        appropriate placeholders inside captions.
    \end{itemize}
\end{itemize}


%% 日本の成果物
\section{Japanese deliverables in \TL}

There are many packages developed specially for Japanese typesetting
in \TL; most of these are installed as part of
\Pkg{collection-langjapanese}.
Only a subset of these useful packages are shown below.
For more details, please refer to individual documentation.

\subsection{Document classes}

Japanese text can be aligned in two directions.
One is horizontal direction, called \emph{yoko-gumi} (横組),
the other is vertical direction, called \emph{tate-gumi} (縦組).
The list below shows \ClsT{\emph{tate} classes in green}
and \ClsYT{\emph{yoko}/\emph{tate} classes in blue} (default is
\emph{yoko}, and the class option \Opt{[tate]} switches to
\emph{tate} mode).

All the major frameworks for Japanese typesetting provide
``standard classes'':
\begin{itemize}
  \item Lua\TeX-ja (\Pkg{luatexja} by Lua\TeX-ja project team)
    \begin{itemize}
      \item Bundle: \Pkg{ltjsclasses}
         (\ClsY{ltjsarticle}, \ClsY{ltjsbook}, \ClsY{ltjsreport})
      \item Bundle: \Pkg{ltjclasses}
         (\ClsY{ltjarticle}, \ClsY{ltjbook}, \ClsY{ltjreport},
          \ClsT{ltjtarticle}, \ClsT{ltjtbook}, \ClsT{ltjtreport})
    \end{itemize}
  \item \upLaTeX\ (\Pkg{uplatex} by \JTDC)
    \begin{itemize}
      \item Bundle: \Pkg{ujclasses}
         (\ClsY{ujarticle}, \ClsY{ujbook}, \ClsY{ujreport},
          \ClsT{utarticle}, \ClsT{utbook}, \ClsT{utreport})*
    \end{itemize}
  \item \pLaTeX\ (\Pkg{platex} by \JTDC)
    \begin{itemize}
      \item Bundle: \Pkg{jclasses}
         (\ClsY{jarticle}, \ClsY{jbook}, \ClsY{jreport},
          \ClsT{tarticle}, \ClsT{tbook}, \ClsT{treport})*
    \end{itemize}
    * Due to historical reasons, \Pkg{ujclasses} and \Pkg{jclasses}
    are left with some inconvenient behaviors. You may need to adjust
    some layout parameters by hand, or use other classes mentioned below.
\end{itemize}

Other major classes designed for Japanese:
\begin{itemize}
  \item Class: \ClsYT{jlreq} (by Noriyuki Abe)
    \begin{itemize}
      \item Supports \LuaLaTeX, \upLaTeX\ and \pLaTeX.
      \item This class aims to implement
        ``Requirements for Japanese Text Layout''
        (JLReq, 日本語組版処理の要件, \cite{w3c-jlreq}).
        The class file and necessary JFM (Japanese font metric) files
        are included.
    \end{itemize}
  \item Bundle: \Pkg{jsclasses} (by Haruhiko Okumura \& \JTDC)
    \begin{itemize}
      \item Class: \ClsY{jsarticle}, \ClsY{jsbook}, \ClsY{jsreport}
      \begin{itemize}
        \item Supports only \upLaTeX\ (with option \Opt{[uplatex]}) and \pLaTeX.
        \item In addition to layout adjustments, these classes use
          newly developed JFM set (jis metrics) to avoid strange
          behaviors of standard \pTeX\ JFM set (min10 etc).
          For more details, please refer to \cite{ajt2008okumura}.
        \item Note: when font size other than 10pt is specified,
          the engine primitive \verb+\mag+ is employed.
          Conflicts with other packages can lead to inconsistent
          results in graphics and/or page size.
          To avoid this, specify \Opt{nomag} or \Opt{nomag*} option.
      \end{itemize}
    \end{itemize}
\end{itemize}

\subsection{Packages}

\begin{itemize}
  \item Package: \Pkg{plautopatch} (by Hironobu Yamashita)
    \begin{itemize}
      \item Supports \upLaTeX\ and \pLaTeX, but
        does no harm on other \LaTeX\ (silently ignored).
      \item (u)\pLaTeX\ users are recommended to load this package
        \emph{always}, to resolve errors and incompatibilities with
        Western \LaTeX\ packages which are not aware of (u)\pLaTeX.
    \end{itemize}
  \item Package: \Pkg{pxchfon} (by Takayuki Yato)
    \begin{itemize}
      \item Supports \upLaTeX\ and \pLaTeX.
      \item Helps users to declate physical fonts embedded in
        dvipdfmx's PDF output. It utilizes
        \verb+\special{pdf:mapline}+ or
        \verb+\special{pdf:mapfile}+ syntaxes.
    \end{itemize}
  \item Package: \Pkg{otf}
    (\Pkg{japanese-otf}, \Pkg{japanese-otf-uptex} by Shuzaburo Saito \& TTK)
    \begin{itemize}
      \item Supports only \upLaTeX\ (with option \Opt{[uplatex]}) and \pLaTeX.
      \item Main feature 1: allows switching between seven different
        typefaces, compared to the standard of two typefaces.
        This function is realized by shipping seven ``logical fonts''
        (= set of a JFM and a virtual font).
      \item Main feature 2: provides commands for outputting
        many characters by specifying Unicode (\verb+\UTF{....}+)
        or Adobe CID (\verb+\CID{....}+), even with \pLaTeX.
        This function is realized by mapping a huge real font from
        lots of smaller subset virtual fonts.
        A series of shorthand commands (\verb+\aj...+) are
        also available.
      \item Lua\TeX-ja provides \Pkg{luatexja-otf} to partially emulate
        this package.
    \end{itemize}
\end{itemize}

\subsection{Other support files}

\begin{itemize}
  \item \Pkg{ptex-fontmaps}
    \begin{itemize}
      \item Provides map files to be used with dvipdfmx and dvips.
      \item To customize Japanese fonts to be used in the output,
        please use the command \code{kanji-config-updmap}.
    \end{itemize}
  \item \Pkg{cjk-gs-integrate}
    \begin{itemize}
      \item Supports easy configuration of Ghostscript for
        CJK (Chinese, Japanese and Korean) fonts, which is
        necessary for processing of PostScript file output from dvips.
    \end{itemize}
\end{itemize}


%% 参考文献
\begin{thebibliography}{99}
  \bibitem{w3c-jlreq}
    W3C Working Group,
    \newblock ``{\em Requirements for Japanese Text Layout}''.\\
    \url{https://www.w3.org/TR/jlreq/?lang=en}
  \bibitem{ajt2008okumura}
    Haruhiko Okumura,
    \newblock ``{\em \pTeX\ and Japanese Typesetting}''.
    \newblock The Asian Journal of \TeX, Volume~2, No.~1, 2008.\\
    \url{http://ajt.ktug.org/2008/0201okumura.pdf}
\end{thebibliography}

\end{document}
