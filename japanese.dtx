% \iffalse meta-comment
%
% Copyright 2012-2016 Javier Bezos and Johannes L. Braams.
% Copyright 1989-2012 Johannes L. Braams and any individual authors
% listed elsewhere in this file.  All rights reserved.
% 
% It may be distributed and/or modified under the
% conditions of the LaTeX Project Public License, either version 1.3
% of this license or (at your option) any later version.
% The latest version of this license is in
%   http://www.latex-project.org/lppl.txt
% and version 1.3 or later is part of all distributions of LaTeX
% version 2003/12/01 or later.
% 
% This work has the LPPL maintenance status "maintained".
% 
% The Current Maintainer of this work is Javier Bezos.
%
% \fi
%
%\iffalse
%    Tell the \LaTeX\ system who we are and write an entry on the
%    transcript.
%<*dtx>
\ProvidesFile{japanese.dtx}
%</dtx>
%<code>\ProvidesLanguage{japanese}
%<*!sample>
%\ProvidesFile{japanese.dtx}
        [2016/12/15 v2.0 Japanese support for babel (ING, texjporg)]
%</!sample>
%\fi
%
%\iffalse
%
%%
%% File 'japanese.dtx' supports the following Babel package:
%%
%% Babel package for LaTeX2e.
%% Copyright (C) 1989-2008 by Johannes Braams,
%%                            TeXniek
%%                            all rights reserved.
%% Copyright (C) 2013-2016 by Johannes Braams 
%%                            TeXniek
%%                            by Javier Bezos
%%                            all rights reserved.
%%
%
%<*filedriver>
\documentclass{jltxdoc}
\GetFileInfo{japanese.dtx}
\setcounter{StandardModuleDepth}{1}
\CodelineNumbered
\MakeShortVerb{\|}
\OnlyDescription
\newcommand*\babel{\textsf{babel}}
\newcommand*\langvar{$\langle \it lang \rangle$}
%\newcommand*\Lopt[1]{\textsf{#1}}
%\newcommand*\file[1]{\texttt{#1}}
\begin{document}
  \DocInput{japanese.dtx}
\end{document}
%</filedriver>
%\fi
% \GetFileInfo{japanese.dtx}
%
% \parindent=1zw
% \title{Babel-Option {\sffamily japanese} {\Large version \fileversion}}
% \author{\copyright\ 1999--2007 ING\\ 2016-- Japanese \TeX\ Development Community}
% \date{\filedate}
% \maketitle
% \baselineskip=14pt
%
% \section*{The Japanese language}
%
% \file{japanese}パッケージは日本語による見出し語と日付を出力するためのマクロを
% 定義しています。\babel のオプションの最後で日本語を指定します。
% \begin{quote}
% |\usepackage[...,japanese]{babel}|
% \end{quote}
%
% バージョンv2.0以降では,p\TeX{}系(p\LaTeX{},up\LaTeX{})に加えて,
% 新しいUnicodeな\TeX{}エンジン(Xe\TeX{},Lua\TeX{})もサポートしました。
% このため,ファイルはUTF-8エンコーディングで保存するようにしてください。
% p\TeX{}系でなく,かつUnicodeな\TeX{}でない場合は最初にエラーを出します。
%
%    \begin{macrocode}
%<*code>
\ifx\kanjiskip\@undefined\ifx\XeTeXversion\@undefined\ifx\directlua\@undefined
  \@latex@error{babel-japanese supports one of the followings:\MessageBreak
                pTeX, upTeX, XeTeX, LuaTeX\MessageBreak
                It seems you are running unsupported engine!}\@ehc
\fi\fi\fi
%    \end{macrocode}
%
%    \begin{macrocode}
\LdfInit\CurrentOption{captions\CurrentOption}
%    \end{macrocode}
% \DescribeMacro{\l@japanese}
% ここでは|\l@japanese|が定義されているか否かを判断し,定義されていれば
% 日本語用ハイフネーションパターンを読み込みます。
% \DescribeMacro{\adddialect}
% しかし,日本語にはハイフネーションパタンが存在しないので
% |\adddialect|に|\l@japanese|を代入し,\file{language.dat}
% で最初に指定した言語(言語番号0,通常は英語)のハイフネーションパターンを
% 使用します。
% 従って,本パッケージを用いて文章ファイルをコンパイルすると次の警告が
% でますが,無視することにします。
% \begin{quote}\small\begin{verbatim}
% Package babel Warning: No hyphenation patterns were loaded for
% (babel)                the language `Japanese'
% (babel)                I will use the patterns loaded for \language=0
%                        instead.
% \end{verbatim}\end{quote}
%
%    \begin{macrocode}
\ifx\l@japanese\@undefined
  \@nopatterns{Japanese}
  \adddialect\l@japanese0\fi
%    \end{macrocode}
%
% \DescribeMacro{\captionsjapanese}
% |\captionsjapanese|マクロはp\LaTeX{}の標準のクラスファイルで使われる
% 見出し語を日本語で出力します。\LaTeX{}の標準のクラスファイルでも動作
% しますが,元が英語用ですので,語順の関係上すべてを日本語化することは
% できません(たとえば、Part 1→第1部とは変更することは不可能です)。
%
% [2016-12-15] 「証明」と「用語集」も日本語化するようにしました。
% |\postpartname| がタイポで |\postpartnam| になっていたのを直しました。
%
% 新しいUnicodeな\TeX{}エンジン(Xe\TeX{},Lua\TeX{})の場合は,
% UTF-8エンコーディングで直接和文文字を記述します。
%    \begin{macrocode}
\ifx\kanjiskip\@undefined
  \def\bbl@jpn@prefacename{前書き}%
  \def\bbl@jpn@refname{参考文献}%
  \def\bbl@jpn@abstractname{概要}%
  \def\bbl@jpn@bibname{参考文献}%
  %\def\bbl@jpn@chaptername{Chapter}%
  \def\bbl@jpn@prechaptername{第}%    -- added
  \def\bbl@jpn@postchaptername{章}%   -- added
  \def\bbl@jpn@presectionname{}%  第  -- added
  \def\bbl@jpn@postsectionname{}% 節  -- added
  \def\bbl@jpn@appendixname{付録}%
  \def\bbl@jpn@contentsname{目次}%
  \def\bbl@jpn@listfigurename{図目次}%
  \def\bbl@jpn@listtablename{表目次}%
  \def\bbl@jpn@indexname{索引}%
  \def\bbl@jpn@figurename{図}%
  \def\bbl@jpn@tablename{表}%
  %\def\bbl@jpn@partname{Part}%
  \def\bbl@jpn@prepartname{第}%       -- added
  \def\bbl@jpn@postpartname{部}%      -- added
  %\def\bbl@jpn@enclname{encl}%     同封物
  %\def\bbl@jpn@ccname{cc}%         Cc
  %\def\bbl@jpn@headtoname{To}%     To(宛先)
  %\def\bbl@jpn@pagename{Page}%     ページ
  %\def\bbl@jpn@seename{see}%       参照
  %\def\bbl@jpn@alsoname{see also}% も参照
  \def\bbl@jpn@proofname{証明}%
  \def\bbl@jpn@glossaryname{用語集}%
%    \end{macrocode}
%
% p\TeX{}系の場合は,UTF-8を読めない可能性があるため,
% |\kansuji| トリックを使って和文文字の代わりにします。
%
%    \begin{macrocode}
\else
\begingroup
  \kansujichar1=\jis"4130\relax % 前
  \kansujichar2=\jis"3D71\relax % 書
  \kansujichar3=\jis"242D\relax % き
  \kansujichar4=\jis"3B32\relax % 参
  \kansujichar5=\jis"394D\relax % 考
  \kansujichar6=\jis"4A38\relax % 文
  \kansujichar7=\jis"3825\relax % 献
  \kansujichar8=\jis"3335\relax % 概
  \kansujichar9=\jis"4D57\relax % 要
  \xdef\bbl@jpn@prefacename{\kansuji123}%
  \xdef\bbl@jpn@refname{\kansuji4567}%
  \xdef\bbl@jpn@abstractname{\kansuji89}%
  \xdef\bbl@jpn@bibname{\kansuji4567}%
  \kansujichar1=\jis"4268\relax % 第
  \kansujichar2=\jis"3E4F\relax % 章
  \kansujichar3=\jis"4061\relax % 節
  \kansujichar4=\jis"4955\relax % 付
  \kansujichar5=\jis"4F3F\relax % 録
  \kansujichar6=\jis"3A77\relax % 索
  \kansujichar7=\jis"307A\relax % 引
  \kansujichar8=\jis"4974\relax % 部
  \xdef\bbl@jpn@prechaptername{\kansuji1}%
  \xdef\bbl@jpn@postchaptername{\kansuji2}%
  \xdef\bbl@jpn@appendixname{\kansuji45}%
  \xdef\bbl@jpn@indexname{\kansuji67}%
  \xdef\bbl@jpn@prepartname{\kansuji1}%
  \xdef\bbl@jpn@postpartname{\kansuji8}%
  \kansujichar1=\jis"4C5C\relax % 目
  \kansujichar2=\jis"3C21\relax % 次
  \kansujichar3=\jis"3F5E\relax % 図
  \kansujichar4=\jis"493D\relax % 表
  \kansujichar5=\jis"3E5A\relax % 証
  \kansujichar6=\jis"4C40\relax % 明
  \kansujichar7=\jis"4D51\relax % 用
  \kansujichar8=\jis"386C\relax % 語
  \kansujichar9=\jis"3D38\relax % 集
  \xdef\bbl@jpn@contentsname{\kansuji12}%
  \xdef\bbl@jpn@listfigurename{\kansuji312}%
  \xdef\bbl@jpn@listtablename{\kansuji412}%
  \xdef\bbl@jpn@figurename{\kansuji3}%
  \xdef\bbl@jpn@tablename{\kansuji4}%
  \xdef\bbl@jpn@proofname{\kansuji56}%
  \xdef\bbl@jpn@glossaryname{\kansuji789}%
\endgroup
\fi
%    \end{macrocode}
%
% 実際の命令にこれらをコピーします。
%
%    \begin{macrocode}
\@namedef{captions\CurrentOption}{%
  \let\prefacename\bbl@jpn@prefacename
  \let\refname\bbl@jpn@refname
  \let\abstractname\bbl@jpn@abstractname
  \let\bibname\bbl@jpn@bibname
  %\def\chaptername{Chapter}%
  \let\prechaptername\bbl@jpn@prechaptername   % -- added
  \let\postchaptername\bbl@jpn@postchaptername % -- added
  \def\presectionname{}%  第  -- added
  \def\postsectionname{}% 節  -- added
  \let\appendixname\bbl@jpn@appendixname
  \let\contentsname\bbl@jpn@contentsname
  \let\listfigurename\bbl@jpn@listfigurename
  \let\listtablename\bbl@jpn@listtablename
  \let\indexname\bbl@jpn@indexname
  \let\figurename\bbl@jpn@figurename
  \let\tablename\bbl@jpn@tablename
  %\def\partname{Part}%
  \let\prepartname\bbl@jpn@prepartname   % -- added
  \let\postpartname\bbl@jpn@postpartname % -- added
  %\def\enclname{encl}%     同封物
  %\def\ccname{cc}%         Cc
  %\def\headtoname{To}%     To(宛先)
  %\def\pagename{Page}%     ページ
  %\def\seename{see}%       参照
  %\def\alsoname{see also}% も参照
  \let\proofname\bbl@jpn@proofname
  \let\glossaryname\bbl@jpn@glossaryname
  }
%    \end{macrocode}
%
% \DescribeMacro{\datejapanese}
% |\datejapanese|マクロは日本語で日付を出力するように |\today|コマンドを
% 再定義します。デフォルトの出力は西暦です。和暦を使用する際は,プリアンブルで 
% |\和暦| を指定するか,本文で |\和暦\today| のように指定します。
%
% フラグの準備,平成の計算。
%    \begin{macrocode}
\newif\ifbbl@jpn@Seireki \bbl@jpn@Seirekitrue
{\advance\year-1988\relax
 \xdef\the@heisei{\the\year}}
%    \end{macrocode}
%
% Unicodeな\TeX{}エンジン(Xe\TeX{},Lua\TeX{})の場合は,
% UTF-8エンコーディングで直接和文文字を記述します。
%    \begin{macrocode}
\ifx\kanjiskip\@undefined
  \def\西暦{\bbl@jpn@Seirekitrue}%
  \def\和暦{\bbl@jpn@Seirekifalse}%
  \def\bbl@jpn@SeirekiToday{%
      \number\year 年%
      \number\month 月%
      \number\day 日}
  \def\bbl@jpn@WarekiToday{%
      平成\the@heisei 年%
      \number\month 月%
      \number\day 日}
%    \end{macrocode}
%
% p\TeX{}系では |\kansuji| トリックで同じ命令を定義します。
%    \begin{macrocode}
\else
\begingroup
  \kansujichar1=\jis"403E\relax % 西
  \kansujichar2=\jis"4F42\relax % 和
  \kansujichar3=\jis"4E71\relax % 暦
  \expandafter\expandafter\expandafter\gdef
  \expandafter\csname\kansuji13\endcsname{\bbl@jpn@Seirekitrue}%
  \expandafter\expandafter\expandafter\gdef
  \expandafter\csname\kansuji23\endcsname{\bbl@jpn@Seirekifalse}%
  \kansujichar1=\jis"472F\relax % 年
  \kansujichar2=\jis"376E\relax % 月
  \kansujichar3=\jis"467C\relax % 日
  \kansujichar4=\jis"4A3F\relax % 平
  \kansujichar5=\jis"402E\relax % 成
  \xdef\bbl@jpn@SeirekiToday{%
    \number\year\expandafter\@firstofone\expandafter{\kansuji1}%
    \number\month\expandafter\@firstofone\expandafter{\kansuji2}%
    \number\day\expandafter\@firstofone\expandafter{\kansuji3}}
  \xdef\bbl@jpn@WarekiToday{%
    \expandafter\@firstofone\expandafter{\kansuji45}%
    \the@heisei\expandafter\@firstofone\expandafter{\kansuji1}%
    \number\month\expandafter\@firstofone\expandafter{\kansuji2}%
    \number\day\expandafter\@firstofone\expandafter{\kansuji3}}
\endgroup
\fi
%    \end{macrocode}
%
% 実際に使用する命令。
%    \begin{macrocode}
\@namedef{date\CurrentOption}{%
  \def\today{%
    \ifbbl@jpn@Seireki
      \bbl@jpn@SeirekiToday
    \else
      \bbl@jpn@WarekiToday
    \fi}}
%    \end{macrocode}
%
%    \begin{macrocode}
\@namedef{extras\CurrentOption}{}
\@namedef{noextras\CurrentOption}{}
\ldf@finish\CurrentOption
%</code>
%    \end{macrocode}
%
% \section*{謝辞}
%
% Babel-Option \Lopt{japanese}の作成に当って,バグフィックスや改良案を
% ご提案いただいた方に感謝します。
% bookworm $<$BYV01204$>$さんから,新しい言語を定義し,それに固有の
% 言語番号を付けるマクロ \verb+\addlanguage+の機能について,詳しい
% 解説をいただきました。本パッケージでは採用していませんが,\babel
% の言語切り替え機能を理解する上でたいへん参考になりました。
% Tony $<$PAG01322$>$さんから,キャプションと日付の定義について
% ご提案をいただきました。
% 大石勝 $<$DZH00446$>$さんから,初版に含まれていた \verb+\ifx\undefined+
% のバグをご指摘いただきました。
%
% \section*{変更履歴}
% \begin{itemize}
% \item 2005年2月:日付の定義を修正しました。
% \item 2007年10月:ZRさんからいただいた詳細なご指摘をもとに修正しました。\\
% \verb|http://oku.edu.mie-u.ac.jp/~okumura/texfaq/qa/48625.html|
% \item 2007年12月:ZRさん,ttkさんからいただいたご指摘を反映しました。
% \item 2016年12月:日本語\TeX{}開発コミュニティが開発を引き継ぎました。
% 以降の変更履歴は本文中に直接書いてあります。
% \end{itemize}
%
% \StopEventually{}
%
%    \begin{macrocode}
%<*sample>
\documentclass{jbook}
\usepackage[german,english,japanese]{babel}
\makeatletter
\def\tbcaption{\def\@captype{table}\caption{キャプションの例}}
\def\fgcaption{\def\@captype{figure}\caption{キャプションの例}}
\makeatother
\def\yes{--- はい。}
\def\no{--- いいえ。}
\def\TEXT{Textverarbeitung mit einem Rechner kann in vielf\"altiger Weise
erfolgen. Eigenschaften und Leistungsf\"ahigkeit sind hierbei weniger
vom jeweiligen Rechnertype, sondern vielmehr vom verwendeten
\textit{Textverarbeitungsprogramm} bestimmt.}
\def\se{\selectlanguage{english}}
\def\sj{\selectlanguage{japanese}}
\def\sg{\selectlanguage{german}}
\setlength{\hoffset}{-13mm}
\setlength{\textwidth}{16cm}
\begin{document}

\chapter{babel}
\section{japaneseパッケージ}
japaneseパッケージは日本語による見出し語と日付を出力するためのマクロを
定義しています。

\fgcaption

\begin{itemize}
\se
\item ここで英語(\texttt{english})に変更します。(languageの値は\the\language)

\TEXT

ここは英語? \iflanguage{english}{\yes}{\no}\par
ここはドイツ語? \iflanguage{german}{\yes}{\no}\par
ここは日本語? \iflanguage{japanese}{\yes}{\no}

※ \verb:\adddialect\l@japanese0: と設定しているため,日本語?も「はい」となります。

\sg
\item ここでドイツ語(\texttt{german})に変更します。(languageの値は\the\language)

\TEXT

ここは英語? \iflanguage{english}{\yes}{\no}\par
ここはドイツ語? \iflanguage{german}{\yes}{\no}\par
ここは日本語? \iflanguage{japanese}{\yes}{\no}

※ ハイフネーションがドイツ語―旧正書法―に切り替わっている点に注目。
なお,新正書法(\texttt{ngerman})では\texttt{Leis-tungs-f\"a-hig-keit}のように分綴します。

\sj
\item ここで日本語(\texttt{japanese})に変更します。(languageの値は\the\language)
\tbcaption
\item \verb:\和暦: は日付の表示をデフォルトの西暦「\today 」から和暦「\和暦\today 」に変更します。
\end{itemize}
\end{document}
%</sample>
%    \end{macrocode>
%
% \Finale
%
\endinput
