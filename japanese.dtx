% \iffalse meta-comment
%
% Copyright 1989-2007 Johannes L. Braams and any individual authors
% listed elsewhere in this file.  All rights reserved.
% 
% It may be distributed and/or modified under the
% conditions of the LaTeX Project Public License, either version 1.3
% of this license or (at your option) any later version.
% The latest version of this license is in
%   http://www.latex-project.org/lppl.txt
% and version 1.3 or later is part of all distributions of LaTeX
% version 2003/12/01 or later.
% 
% This work has the LPPL maintenance status "maintained".
% 
% The Current Maintainer of this work is Johannes Braams.
% 
% \fi
%\iffalse
%    Tell the \LaTeX\ system who we are and write an entry on the
%    transcript.
%<*dtx>
\ProvidesFile{japanese.dtx}
%</dtx>
%<code>\ProvidesLanguage{japanese}
%\fi
%\ProvidesFile{japanese.dtx}
%        [2007/12/08 v1.3 Japanese support]
%\iffalse
%% File 'japanese.dtx'
%% Babel package for LaTeX version 2e
%% Copyright (C) 1989 - 2007
%%           by Johannes Braams, TeXniek
%
%<*filedriver>
\documentclass{ltxdoc}
\setcounter{StandardModuleDepth}{1}
\CodelineNumbered
\MakeShortVerb{\|}
\OnlyDescription
\newcommand*\babel{\textsf{babel}}
\newcommand*\langvar{$\langle \it lang \rangle$}
\newcommand*\Lopt[1]{\textsf{#1}}
\newcommand*\file[1]{\texttt{#1}}
\begin{document}
 \DocInput{japanese.dtx}
\end{document}
%</filedriver>
%\fi
% \GetFileInfo{japanese.dtx}
%
% \parindent=1zw
% \title{Babel-Option {\sffamily japanese} \\ {\Large version 1.3}}
% \author{\copyright\ 1999--2007 ING}
% \date{}
% \maketitle
% \baselineskip=14pt
%
%  \section*{The Japanese language}
%
% japaneseパッケージは日本語による見出し語と日付を出力するためのマクロを
% 定義しています。\babel のオプションの最後で日本語を指定します。
% \begin{quote}
% |\usepackage[...,japanese]{babel}| 
% \end{quote}
%
%    \begin{macrocode}
%<*code>
\LdfInit{japanese}\captionsjapanese
%    \end{macrocode}
% \DescribeMacro{\l@japanese}
% ここでは|\l@japanese|が定義されているか否かを判断し,定義されていれば
% 日本語用ハイフネーションパターンを読み込みます。
% \DescribeMacro{\adddialect}
% しかし,日本語にはハイフネーションパタンが存在しないので
% |\adddialect|に|\l@japanese|を代入し,\file{language.dat}
% で最初に指定した言語(言語番号0,通常は英語)のハイフネーションパターンを
% 使用します。
% 従って,本パッケージを用いて文章ファイルをコンパイルすると次の警告が
% でますが,無視することにします。
% \begin{quote}\small\begin{verbatim}
% Package babel Warning: No hyphenation patterns were loaded for
% (babel)                the language `Japanese'
% (babel)                I will use the patterns loaded for \language=0
%                        instead.
% \end{verbatim}\end{quote}
%
%    \begin{macrocode}
\ifx\l@japanese\@undefined
  \@nopatterns{Japanese}
  \adddialect\l@japanese0\fi
%    \end{macrocode}
%
% \DescribeMacro{\captionsjapanese}
% |\captionsjapanese|マクロはp\LaTeX{}の標準のクラスファイルで使われる
% 見出し語を日本語で出力します。
%
%    \begin{macrocode}
\addto\captionsjapanese{%
  \def\prepartname{第}%
  \def\postpartnam{部}%
  \def\prechaptername{第}%
  \def\postchaptername{章}%
  \def\presectionname{}%  第
  \def\postsectionname{}% 節
  \def\prefacename{前書き}%
  \def\refname{参考文献}%
  \def\bibname{参考文献}%
  \def\abstractname{概要}%
  \def\appendixname{付録}%
  \def\contentsname{目次}%
  \def\listfigurename{図目次}%
  \def\listtablename{表目次}%
  \def\indexname{索引}%
  \def\figurename{図}%
  \def\tablename{表}%
  }
%    \end{macrocode}
% \DescribeMacro{\datejapanese}
% |\datejapanese|マクロは日本語で日付を出力するように |\today|コマンドを
% 再定義します。デフォルトの出力は西暦です。和暦を使用する際は,プリアンブルで 
% |\和暦| を指定するか,本文で |\和暦\today| のように指定します。
%    \begin{macrocode}
\newif\if西暦 \西暦true%
\def\西暦{\西暦true}%
\def\和暦{\西暦false}%
{\advance\year-1988\relax
 \xdef\the@heisei{\the\year}}
\def\datejapanese{%
  \def\today{%
    \if西暦%
      \number\year 年% 
      \number\month 月% 
      \number\day 日%
    \else
      平成\the@heisei 年%
      \number\month 月%
      \number\day 日%
    \fi}}
\ldf@finish{japanese}
%</code>
%    \end{macrocode}
%
% \section*{謝辞}
%
% Babel-Option \Lopt{japanese}の作成に当って,バグフィックスや改良案を
% ご提案いただいた方に感謝します。
% bookworm $<$BYV01204$>$さんから,新しい言語を定義し,それに固有の
% 言語番号を付けるマクロ \verb+\addlanguage+の機能について,詳しい
% 解説をいただきました。本パッケージでは採用していませんが,\babel
% の言語切り替え機能を理解する上でたいへん参考になりました。
% Tony $<$PAG01322$>$さんから,キャプションと日付の定義について
% ご提案をいただきました。
% 大石勝 $<$DZH00446$>$さんから,初版に含まれていた \verb+\ifx\undefined+
% のバグをご指摘いただきました。
%
% \section*{変更履歴}
% \begin{itemize}
% \item 2005年2月:日付の定義を修正しました。
% \item 2007年10月:ZRさんからいただいた詳細なご指摘をもとに修正しました。\\
% \verb|http://oku.edu.mie-u.ac.jp/~okumura/texfaq/qa/48625.html|
% \item 2007年12月:ZRさん,ttkさんからいただいたご指摘を反映しました。
% \end{itemize}
%
% \StopEventually{}
%
%    \begin{macrocode}
%<*sample>
\documentclass{jbook}
\usepackage[german,english,japanese]{babel}
\makeatletter
\def\tbcaption{\def\@captype{table}\caption{キャプションの例}}
\def\fgcaption{\def\@captype{figure}\caption{キャプションの例}}
\makeatother
\def\yes{--- はい。}
\def\no{--- いいえ。}
\def\TEXT{Textverarbeitung mit einem Rechner kann in vielf\"altiger Weise
erfolgen. Eigenschaften und Leistungsf\"ahigkeit sind hierbei weniger
vom jeweiligen Rechnertype, sondern vielmehr vom verwendeten
\textit{Textverarbeitungsprogramm} bestimmt.}
\def\se{\selectlanguage{english}}
\def\sj{\selectlanguage{japanese}}
\def\sg{\selectlanguage{german}}
\setlength{\hoffset}{-13mm}
\setlength{\textwidth}{16cm}
\begin{document}

\chapter{babel}
\section{japaneseパッケージ}
japaneseパッケージは日本語による見出し語と日付を出力するためのマクロを
定義しています。

\fgcaption

\begin{itemize}
\se
\item ここで英語(\texttt{english})に変更します。(languageの値は\the\language)

\TEXT

ここは英語? \iflanguage{english}{\yes}{\no}\par
ここはドイツ語? \iflanguage{german}{\yes}{\no}\par
ここは日本語? \iflanguage{japanese}{\yes}{\no}

※ \verb:\adddialect\l@japanese0: と設定しているため,日本語?も「はい」となります。

\sg
\item ここでドイツ語(\texttt{german})に変更します。(languageの値は\the\language)

\TEXT

ここは英語? \iflanguage{english}{\yes}{\no}\par
ここはドイツ語? \iflanguage{german}{\yes}{\no}\par
ここは日本語? \iflanguage{japanese}{\yes}{\no}

※ ハイフネーションがドイツ語―旧正書法―に切り替わっている点に注目。
なお,新正書法(\texttt{ngerman})では\texttt{Leis-tungs-f\"a-hig-keit}のように分綴します。

\sj
\item ここで日本語(\texttt{japanese})に変更します。(languageの値は\the\language)
\tbcaption
\item \verb:\和暦: は日付の表示をデフォルトの西暦「\today 」から和暦「\和暦\today 」に変更します。
\end{itemize}
\end{document}
%</sample>
%    \end{macrocode>
% \Finale
%%
%% \CharacterTable
%%  {Upper-case    \A\B\C\D\E\F\G\H\I\J\K\L\M\N\O\P\Q\R\S\T\U\V\W\X\Y\Z
%%   Lower-case    \a\b\c\d\e\f\g\h\i\j\k\l\m\n\o\p\q\r\s\t\u\v\w\x\y\z
%%   Digits        \0\1\2\3\4\5\6\7\8\9
%%   Exclamation   \!     Double quote  \"     Hash (number) \#
%%   Dollar        \$     Percent       \%     Ampersand     \&
%%   Acute accent  \'     Left paren    \(     Right paren   \)
%%   Asterisk      \*     Plus          \+     Comma         \,
%%   Minus         \-     Point         \.     Solidus       \/
%%   Colon         \:     Semicolon     \;     Less than     \<
%%   Equals        \=     Greater than  \>     Question mark \?
%%   Commercial at \@     Left bracket  \[     Backslash     \\
%%   Right bracket \]     Circumflex    \^     Underscore    \_
%%   Grave accent  \`     Left brace    \{     Vertical bar  \|
%%   Right brace   \}     Tilde         \~}
%%
\endinput
